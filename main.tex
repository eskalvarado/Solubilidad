\documentclass[a4paper,12pt]{article}

\usepackage{amsmath}
\usepackage{graphicx}
\usepackage[colorinlistoftodos]{todonotes}

\usepackage{a4wide}
\usepackage{amsfonts}
\usepackage{amssymb}
\usepackage{setspace}
%\usepackage{makeidx}
\usepackage{multirow}
\usepackage{indentfirst}
\usepackage{float}
\usepackage{caption}
\usepackage{textcomp}
\usepackage{array}
\usepackage{bigstrut}

\usepackage{chemfig}
%%%%%%%%%%%%%%%%%%%%%%%%%%%%%%%%%%%%%%%%%%%%%%%%%%%%%%%%%%%%%%%%%%%%%%%%%%%%%
\doublespace
\rmfamily
\setcounter{tocdepth}{1}
%%%%%%%%%%%%%%%%%%%%%%%%%%%%%%%%%%%%%%%%%%%%%%%%%%%%%%%%%%%%%%%%%%%%%%%%%%%%%
\begin{document}

\begin{titlepage}
	\centering
	{\Large UNIVERSIDAD DE PUERTO RICO}
	
	{\Large RECINTO UNIVERSITARIO DE MAYAGUEZ}
	
	{\Large DEPARTAMENTO DE QUIMICA}
	\vspace{1cm}
	
	{\large QUIM 4101-010} 
	
	{\large LABORATORIO DE QUIMICA FIS\'ICA I}
	\vfill
	
	{\large Experimento: Solubilidad como func\'ion de temperatura y calor diferencial de disolucion}
	\vfill
	
	Profesor: Astrid Cruz \hspace{6cm} Nombres: Edvin Alvarado Velez
	
	March 20,2016 \hspace{8cm} Edgard Lebron			
\end{titlepage}
%%%%%%%%%%%%%%%%%%%%%%%%%%%%%%%%%%%%%%%%%%%%%%%%%%%%%%%%%%%%%%%%%%%%%%%%%%%%%%%%%%%%%%%%%%%%%%%%%%%%%
\section{Objetivo}

	Determinar la solubilidad de \'acido ox\'alico en agua a diferentes temperaturas y calcular el calor diferencial de disoluci\'on.

\section{Data experimental y calculos}

\begin{table}[hbtp]
	\centering
		\begin{tabular}{|l|l|} \hline 
		$V_{HCl}$ & 20 mL \\ \hline 
		$M_{HCl,\ valorado}$ & 0.1061 g/mol \\ \hline 
		$V_{NaOH,\ consumed}$ & 3.53 mL \\ \hline 
		$M_{NaOH,\ exp}$ & 0.601133 M \\ \hline 
		${Mm}_{ac.oxalico}$ & 90.03488 g/mol \\ \hline 
	\end{tabular}
\end{table}

\vspace{1cm}

\[M_{NaOH}=\ \frac{M_{HCl}\times V_{HCl}}{V_{NaOH}}=\frac{0.1061\times 20.0}{3.53}=0.601133\ M\] 
\[n_{ac.\ oxalico}=2\times V_{NaOH}\times M_{NaOH}=2\times \frac{10.50}{1000}\times 0.601133=1.262\times {10}^{-2}\ mol\] 
\[m_{ac.oxalico}=n_{ac.oxalico}\times {Mm}_{ac.oxalico}\] 

% Table generated by Excel2LaTeX from sheet 'Sheet1'
\begin{table}[htbp]
  %\centering
  \caption{Add caption}
    \begin{tabular}{|c|c|p{0.6in}|p{0.7in}|p{0.7in}|p{0.8in}|p{0.8in}|c|}
    \hline
    muestra & temp \textdegree C & peso de la soln (g) & V NaOH (mL) & peso del agua (g) & peso de ac. oxalico (g) & moles de ac. oxalico (mol) & molalidad (m) \bigstrut[t]\\
    1     & 6.0   & 5.0828 & 10.50 & 3.9462 & 1.13658 & 1.262E-02 & 3.199E-03 \\
    2     & 6.0   & 5.0511 & 10.70 & 3.8929 & 1.15823 & 1.286E-02 & 3.305E-03 \\
    3     & 15.9  & 5.0328 & 10.45 & 3.9016 & 1.13117 & 1.256E-02 & 3.220E-03 \\
    4     & 15.9  & 5.0549 & 10.40 & 3.9291 & 1.12576 & 1.250E-02 & 3.182E-03 \\
    5     & 26.0  & 5.0693 & 10.40 & 3.9435 & 1.12576 & 1.250E-02 & 3.171E-03 \\
    6     & 26.0  & 5.0674 & 10.45 & 3.9362 & 1.13117 & 1.256E-02 & 3.192E-03 \\
    7     & 35.0  & 5.0487 & 10.50 & 3.9121 & 1.13658 & 1.262E-02 & 3.227E-03 \\
    8     & 35.0  & 5.0576 & 10.41 & 3.9308 & 1.12684 & 1.252E-02 & 3.184E-03 \bigstrut[b]\\
    \hline
    \end{tabular}%
  \label{tab:addlabel}%
\end{table}%





\section{Datos para la construccion de la grafica}

% Table generated by Excel2LaTeX from sheet 'Sheet1'
\begin{table}[hbtp]
  \centering
  \caption{Add caption}
    \begin{tabular}{|c|c|c|c|}
    \hline
    Temp. (K) & molalidad (m) & 1/T & $ln(m)$ \bigstrut\\
    \hline
    279.15 & \multicolumn{1}{c|}{3.199E-03} & 3.582E-03 & -5.74492933 \bigstrut[t]\\
    279.15 & \multicolumn{1}{c|}{3.305E-03} & 3.582E-03 & -5.71244957 \\
    289.05 & \multicolumn{1}{c|}{3.220E-03} & 3.460E-03 & -5.73833945 \\
    289.05 & \multicolumn{1}{c|}{3.182E-03} & 3.460E-03 & -5.75016236 \\
    299.15 & \multicolumn{1}{c|}{3.171E-03} & 3.343E-03 & -5.75382059 \\
    299.15 & \multicolumn{1}{c|}{3.192E-03} & 3.343E-03 & -5.74716845 \\
    308.15 & \multicolumn{1}{c|}{3.227E-03} & 3.245E-03 & -5.7362506 \\
    308.15 & \multicolumn{1}{c|}{3.184E-03} & 3.245E-03 & -5.74961288 \bigstrut[b]\\
    \hline
    \end{tabular}%
  \label{tab:addlabel}%
\end{table}%

Grafique vs 1/T y obtenga el calor diferencial de disoluci\'on a cada una de las temperaturas estudiadas. Complete la siguiente tabla:

\begin{figure}[h]
	\centering
	\includegraphics[scale=1]{1.png}
\end{figure}

% Table generated by Excel2LaTeX from sheet 'Sheet1'
\begin{table}[htbp]
  \centering
  \caption{Add caption}
    \begin{tabular}{|c|c|c|}
    \hline
    Temp (K) & molalidad (m) & $\Delta H$ \bigstrut\\
    \hline
    279.15 & 3.262E-03 & -8.3408E+07 \bigstrut[t]\\
    289.05 & 3.176E-03 & -8.0561E+07 \\
    299.15 & 3.209E-03 & -7.7850E+07 \\
    308.15 & 3.184E-03 & -7.5583E+07 \bigstrut[b]\\
    \hline
    \end{tabular}%
  \label{tab:addlabel}%
\end{table}%



\begin{align*}
	f'(x) &= -2.7 \times 10^9 x^2 + 2 \times 10^7 x - 34.759\\
	\Delta H_{diff} &= -Rf'(\frac{1}{T_i})\\
	\Delta H_{diff,\ 6.0\ C} &= -8.3145f'\large(\frac{1}{298.15\ k}\large) = -8.3145 \times 10^7\ J
\end{align*}



\section{Datos Teoricos}

% Table generated by Excel2LaTeX from sheet 'Sheet2'
\begin{table}[htbp]
  \centering
  \caption{Add caption}
    \begin{tabular}{|c|c|c|c|}
    \hline
    \multicolumn{1}{|c|}{T} & m     & 1/T   & ln(m) \bigstrut\\
    \hline
    280.15 & 6.68E-04 & 3.57E-03 & -7.31074 \bigstrut[t]\\
    285.15 & 8.88E-04 & 3.51E-03 & -7.02701 \\
    299.15 & 1.32E-03 & 3.34E-03 & -6.62971 \\
    318.15 & 1.85E-03 & 3.14E-03 & -6.29522 \\
    323.45 & 2.29E-03 & 3.09E-03 & -6.08029 \\
    338.15 & 3.01E-03 & 2.96E-03 & -5.80629 \\
    351.25 & 4.39E-03 & 2.85E-03 & -5.4275 \bigstrut[b]\\
    \hline
    \end{tabular}%
  \label{tab:addlabel}%
\end{table}%

\begin{figure}
	\centering
	\includegraphics[scale=1]{2.png}
\end{figure}

\todo{no se pudo conseguir datos iguales so cogi los data points mas cercas}

% Table generated by Excel2LaTeX from sheet 'Sheet2'
\begin{table}[htbp]
  \centering
  \caption{Add caption}
    \begin{tabular}{|r|c|c|}
    \hline
    \multicolumn{1}{|c|}{T(k)} & 1/T   & $\Delta H$(J) \bigstrut\\
    \hline
    280.15 & 3.570E-03 & -7.373E+06 \bigstrut[t]\\
    285.15 & 3.507E-03 & -7.167E+06 \\
    299.15 & 3.343E-03 & -6.626E+06 \\
    318.15 & 3.143E-03 & -5.969E+06 \\
    323.45 & 3.092E-03 & -5.799E+06 \\
    338.15 & 2.957E-03 & -5.356E+06 \\
    351.25 & 2.847E-03 & -4.993E+06 \bigstrut[b]\\
    \hline
    \end{tabular}%
  \label{tab:addlabel}%
\end{table}%



\section{Discusion de resultados}

\section{Conclusion}

\section*{References}

\begin{itemize}
	\item  "Oxalic Acid {\textbar} C2H2O4 - Pubchem". \textit{Pubchem.ncbi.nlm.nih.gov}. N.p., 2016. Web. 8 Mar. 2016. - https://pubchem.ncbi.nlm.nih.gov/compound/oxalic\_acid\#section=Top

	\item  "CHEMISTRY 122 - LAB: CORRECTIONS, SUGGESTIONS, PRELAB AND REPORT QUESTIONS - EXP 15". \textit{Chemistry.osu.edu}. N.p., 2016. Web. 18 Mar. 2016. - https://chemistry.osu.edu/$\sim$rzellmer/chem122/lab/exp15.pdf
\end{itemize}

\end{document}
